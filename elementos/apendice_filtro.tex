\chapter{Projeto de filtro passa-baixas butterworth de segunda ordem}
\label{apend:1}
Para o projeto do filtro, foi considerada uma frequência de corte de 15 Hz e um o período de amostragem de 0,01 segundos. Em particular, projetou-se um filtro digital do tipo Butterwoth de segunda ordem. Especificamente, o seguinte filtro foi utilizado:

\begin{equation}
    H(z) = \frac{0,1311z^2-0,2596z+0,1285}{z^2-2.007z+1,008}
    \label{transformada_filtro}
\end{equation}

Por meio da transformada Z inversa, torna-se possível obter a equação para implementação no microcontrolador, sendo $y$ o sinal e $y_f$ o sinal filtrado para um instante de tempo discreto.

\begin{equation}
    y[k]= 0,7478y_f[k-1] -0,2722y_f[k-2] +0,1311y[k] +0,2622y[k-1]+0.1311y[k-2]
    \label{filtro}
\end{equation}

Para fins de ilustração, a Figura \ref{fig:resp_filtr} mostra a comparação do sinal do \textit{encoder} do motor antes e depois do filtro. Pode-se observar que ocorre uma diminuição considerável do ruído na leitura do sinal, mas mesmo assim não é gerado um atraso significativo na leitura do sensor. Sendo assim, considera-se aceitável o projeto de filtro para o trabalho em questão.

\begin{figure}[H]
    \centering
    \includegraphics[width=0.8\linewidth]{figuras/filtro_12V.eps}
    \caption[Comparação do sinal do \textit{encoder} antes e depois do filtro passa-baixas]{Comparação do sinal do \textit{encoder} antes e depois do filtro passa-baixas.}
    \label{fig:resp_filtr}
\end{figure}